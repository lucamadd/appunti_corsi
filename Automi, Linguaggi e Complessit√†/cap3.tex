\chapter{Gerarchia di Chomsky}
Introduciamo adesso una definizione di grammatica che è più generale di quella data in precedenza (grammatica context-free).
Questo concetto eredita molte caratteristiche dai sistemi di Thue, di Post, di riscrittura,... e fu introdotto da Noam Chomsky allo scopo di formalizzare le lingue naturali e facilitare la traduzione automatica.
Vedremo che tali grammatiche, nella loro forma più generale, sono Turing-equivalenti, nel senso che gli insiemi di stringhe che generano sono Turing-riconoscibili.
Poiché gli informatici sono interessati a usare le grammatiche per descrivere quelli che sono i programmi “ben formati”, ci siamo soffermati e ci soffermeremo ancora sulle grammatiche context-free, in modo da garantirci sia la decidibilità che l’efficienza.

\vspace{5mm}

Una grammatica (di tipo 0 o a struttura di frase) $G$, è una quadrupla $G=(V, T, P, S)$ dove:
\begin{itemize}
    \item V: Insieme finito di variabili (dette anche non terminali o categorie sintattiche). Ogni variabile rappresenta un linguaggio.
    \item T: Insieme finito di simboli terminali (o alfabeto dei terminali). Ė l'alfabeto delle parole del linguaggio da definire. $T \cap V=\emptyset$.
    \item $P$ : Insieme finito delle produzioni. Ogni produzione ha la forma $\alpha \rightarrow \beta$, dove $\alpha \in(V \cup T)^{+} e \beta \in(V \cup T)^{*}$.
    \item S: Una variabile, detta simbolo iniziale (o start symbol), che rappresenta il linguaggio definito dalla grammatica.
\end{itemize}

La differenza consiste nella definizione di produzione.
Le definizioni di derivazione diretta, derivazione e linguaggio generato sono le stesse.

\subsubsection{Derivazione}
Sia $G=(V, T, P, S)$ una grammatica.
Sia $\alpha \rightarrow \beta \in P, \gamma_{1}, \gamma_{2} \in(V \cup T)^{*}$
Allora diremo che
$\gamma_{1} \alpha \gamma_{2}$ deriva (direttamente) $\gamma_{1} \beta \gamma_{2}$ in $G$ e scriveremo
$$
\gamma_{1} \alpha \gamma_{2} \underset{G}{\Rightarrow} \gamma_{1} \beta \gamma_{2}
$$
o, se G è chiara dal contesto, semplicemente
$$
\gamma_{1} \alpha \gamma_{2} \Rightarrow \gamma_{1} \beta \gamma_{2}
$$

\vspace{5mm}

Sia $G=(V, T, P, S)$ una grammatica, siano $\alpha, \beta \in(V \cup T)^{*}$ Diremo che $\alpha$ deriva $\beta$ in $G, \alpha \underset{G}{\stackrel{*}{\Rightarrow}} \beta$, se esistono parole $\gamma_{1}, \ldots, \gamma_{n}, n \geq 1$, tali cheù
\begin{enumerate}
    \item $\alpha=\gamma_{1}$,
    \item $\beta=\gamma_{n}$,
    \item Per $i=1,2, \ldots, n-1$, si ha $\gamma_{i} \underset{G}{\stackrel{}{\Rightarrow}} \gamma_{i+1}$
\end{enumerate}
In questo caso diremo anche che $\alpha$ deriva $\beta$ in $n-1$ passi. Ovviamente $G$ è omessa se è chiara dal contesto.

\subsubsection{Linguaggio di una grammatica}
Sia $G=(V, T, P, S)$ una grammatica.

Il linguaggio di G è
$$
L(G)=\{w \in T^{*} \mid S \underset{G}{\stackrel{*}{\Rightarrow}} w\}
$$

Nota: Le parole di $L(G)$ sono tutte e sole le stringhe di caratteri terminali per le quali esiste una derivazione dal simbolo iniziale.

\subsubsection{Gerarchia di Chomsky}
Le produzioni di una grammatica di tipo 0 non hanno alcun vincolo, se non quello di avere almeno un simbolo nella loro parte sinistra.
Adesso porremo dei vincoli sulle produzioni, e vedremo che quanto più questi vincoli sono stringenti, tanto minore è il potere espressivo delle grammatiche in questione.
Cioè i vincoli riducono la capacità computazionale del modello.
Otterremo quindi una gerarchia di grammatiche al variare dei vincoli e in corrispondenza una gerarchia di linguaggi.
Vedremo poi che a tale gerarchia corrisponde una gerarchia di automi.
In base alla forma delle produzioni, classificheremo le grammatiche in quattro tipi, indicati brevemente come tipo 0, 1, 2 e 3.

\vspace{5mm}

Una grammatica $G=(V, T, P, S)$ è una grammatica di tipo 0 se ogni $\alpha \rightarrow \beta \in P$ è tale che $\alpha \in(V \cup T)^{+}$ e $ \beta \in(V \cup T)^{*}$.

\vspace{5mm}

Una grammatica $G=(V, T, P, S)$ è una grammatica di tipo 1 o dipendente dal contesto se ogni $\alpha \rightarrow \beta \in P$ è tale che $\alpha \in(V \cup T)^{+}, \beta \in(V \cup T)^{+} e$ il lato destro di ogni produzione ha lunghezza almeno pari al lato sinistro
$$
\forall \alpha \rightarrow \beta \in P \quad |\beta| \geq |\alpha|
$$

Il nome di "grammatica dipendente dal contesto" deriva dal fatto che per ogni $G$ di tipo 1 , esiste $G^{\prime}$ tale che $L(G)=L\left(G^{\prime}\right)$ e in cui le produzioni hanno la forma
$$
\alpha_{1} A \alpha_{2} \rightarrow \alpha_{1} \beta \alpha_{2}
$$
dove $A$ è una variabile, $\alpha_{1}, \alpha_{2}, \beta$ sono stringhe di variabili e terminali e $\beta \neq \epsilon$. (Quindi anche $G^{\prime}$ è di tipo 1).

Cioè è possibile sostituire $A$ con $\beta$ ma solo se $A$ appare nella stringa in un contesto particolare.
Alcuni autori adottano come definizione quella in cui le produzioni hanno questo secondo vincolo. Ma la prima formulazione ha vantaggi che scopriremo.

\vspace{5mm}

Una grammatica $G=(V, T, P, S)$ è una grammatica di tipo 2 se ogni $\alpha \rightarrow \beta \in P$ è tale che $\alpha \in V e \beta \in(V \cup T)^{+}$cioè il lato sinistro di ogni produzione $A \rightarrow \beta$ è una variabile $A$ e il lato destro $\beta$ è una stringa non vuota di variabili e terminali.

Possiamo dire che questa definizione coincide con quella data di grammatica context-free?

\vspace{5mm}

Definizione
Una grammatica $G=(V, T, P, S)$ è una grammatica di tipo 3 o regolare se ogni $\alpha \rightarrow \beta \in P$ è tale che $\alpha \in V$ e $\beta \in T \cup(T \cdot V)$, cioè se ogni produzione è della forma $A \rightarrow a B$ o $A \rightarrow a$, dove $A, B$ sono variabili e a è un terminale.

Nota. Alcuni autori distinguono le grammatiche $G=(V, T, P, S)$ di tipo tre in
\begin{itemize}
    \item grammatica lineare destra se ogni $\alpha \rightarrow \beta \in P$ è tale che $\alpha \in V$ e $\beta \in T \cup(T \cdot V)$, cioè se ogni produzione è della forma $A \rightarrow a B $ o $A \rightarrow a$, dove $A, B$ sono variabili e a è un terminale.
    \item grammatica lineare sinistra se ogni $\alpha \rightarrow \beta \in P$ è tale che $\alpha \in V$ e $\beta \in T \cup(V \cdot T)$, cioè se ogni produzione è della forma $A \rightarrow B a $ o $ A \rightarrow a$, dove $A, B$ sono variabili e a è un terminale.
Non faremo questa distinzione ma si può intuire che per ogni grammatica lineare destra (risp. sinistra) $G$ esiste una grammatica $G^{\prime}$ lineare sinistra (risp. destra) tale che $L(G)=L\left(G^{\prime}\right)$
\end{itemize}

Notiamo che per $i=1,2,3$, le grammatiche di tipo $i$ sono un sottoinsieme delle grammatiche di tipo $i-1$.
\begin{itemize}
    \item Se $G$ è una grammatica di tipo 3 , allora $G$ è anche di tipo 2
    \item Se $G$ è una grammatica di tipo 2, allora $G$ è anche di tipo 1
    \item Se $G$ è una grammatica di tipo 1, allora $G$ è anche di tipo 0
\end{itemize}
Più formalmente possiamo enunciare la seguente proprietà, facilmente dimostrabile.

\subsubsection{Teorema (Gerarchia di grammatiche)}
La classe delle grammatiche di tipo i include strettamente quella delle grammatiche di tipo $i + 1, 0 \leq i \leq 2$.
\begin{itemize}
    \item La grammatica definita dalle produzioni
$$
\begin{array}{rl}
S \rightarrow 0 A B & B \rightarrow 01 \\
1 B \rightarrow 0 & B \rightarrow S A
\end{array} A \begin{gathered}
A 1 \rightarrow S B 1 \\
A 0 \rightarrow S 0 B
\end{gathered}
$$
e di tipo 0 ma non di tipo $1 .$
    \item  La grammatica definita dalle produzioni
$$
\begin{aligned}
&S \rightarrow a S B C \mid a b c \\
&c B \rightarrow B c \\
&b B \rightarrow b b
\end{aligned}
$$
e di tipo 1 ma non di tipo 2 .
    \item  La grammatica definita dalle produzioni
$$
S \rightarrow a S b \mid a b
$$
è di tipo 2 ma non di tipo $3 .$
\end{itemize}

\subsubsection{il problema della parola vuota}
Guardando le definizioni ci accorgiamo che la definizione di grammatica di tipo 2 non corrisponde alla definizione di grammatica context-free che abbiamo dato.

In generale, le grammatiche di tipo $i, i \geq 1$, non possono generare la parola vuota. Questo non è un problema.

Ricordiamo una definizione data.

Due grammatiche $G, G^{\prime}$ sono equivalenti se generano lo stesso linguaggio, cioè $L(G)=L\left(G^{\prime}\right)$.

\vspace{5mm}

Una soluzione "ad hoc" per le grammatiche di tipo 2
Conveniamo di chiamare una grammatica $G=(V, T, P, S)$
\begin{itemize}
    \item context-free se ogni produzione in $P$ ha la forma $A \rightarrow \beta$, con $A \in V$ e $\beta \in(V \cup T)^{*}$;
    \item di tipo 2 se ogni produzione in $P$ ha la forma $A \rightarrow \beta$, con $A \in V \in \beta \in(V \cup T)^{+}$
\end{itemize}

Vogliamo mostrare che un linguaggio è generato da una grammatica context-free se e solo se, a parte eventualmente la parola vuota, è generato da una grammatica di tipo $2 .$

Una grammatica di tipo 2 è un caso particolare di grammatica
context-free. Quindi se un linguaggio L è generato da una
grammatica di tipo 2, allora L è generato da una grammatica
context-free.

Viceversa, data $G=(V, T, P, S)$ CFG che genera $L(G)$ è possibile ottenere $G^{\prime}$ tale che
\begin{enumerate}
    \item $G^{\prime}$ non ha $\epsilon$-produzioni, cioè ogni produzione in $G^{\prime}$ ha la forma $A \rightarrow \beta$, dove $A$ è una variabile e $\beta$ è una stringa non vuota di variabili e terminali. (Quindi $G^{\prime}$ è di tipo 2);
    \item $G^{\prime}$ genera $L(G) \backslash\{\epsilon\}$, cioè $L\left(G^{\prime}\right)=L(G) \backslash\{\epsilon\}$.
Omettiamo la prova. La prova del punto 1 è costruttiva. Ovviamente se $\epsilon \notin L(G)$ allora $L\left(G^{\prime}\right)=L(G)$.
\end{enumerate}

- Nota Se $G=(V, T, P, S)$ è una grammatica di tipo 2, la grammatica $G^{\prime}=\left(V^{\prime}, T, P^{\prime}, S^{\prime}\right)$, dove $S^{\prime} \notin V, V^{\prime}=V \cup\left\{S^{\prime}\right\}$ e $P^{\prime}=P \cup\left\{S^{\prime} \rightarrow S, S^{\prime} \rightarrow \epsilon\right\}$ genera $L(G) \cup\{\epsilon\}$, cioè
$$
L\left(G^{\prime}\right)=L(G) \cup\{\epsilon\}
$$

- Esempio: Sia G la grammatica definita da
$$
S \rightarrow a S b \mid a b
$$
Sia $G^{\prime}$ la grammatica definita da
$$
S^{\prime} \rightarrow S|\epsilon, \quad S \rightarrow a S b| a b
$$
$$
\begin{aligned}
&L(G)=\left\{a^{n} b^{n} \mid n \geq 1\right\}, \\
&L\left(G^{\prime}\right)=\left\{a^{n} b^{n} \mid n \geq 0\right\}=L(G) \cup\{\epsilon\} .
\end{aligned}
$$

Nota

Esiste una soluzione "più elegante" applicabile a tutti e tre i tipi di grammatiche.

Risultato preliminare)

Data $G$ di tipo $i, i=1,2,3$, possiamo sempre trovare $G^{\prime}$ dello stesso tipo ed equivalente a $G$ tale che il simbolo iniziale di $G^{\prime}$ non appaia alla destra di nessuna produzione.

Sia $G=(V, T, P, S)$ una grammatica di tipo $i, i=1,2,3$.
Sia $G^{\prime}=\left(V^{\prime}, T, P^{\prime}, S^{\prime}\right)$, con $S^{\prime} \notin V, V^{\prime}=V \cup\left\{S^{\prime}\right\}$ e dove $P^{\prime}$ è
l'insieme delle produzioni in P più tutte le produzioni $S^{\prime} \rightarrow \alpha$, dove $S \rightarrow \alpha \in P$, cioè
$$
P^{\prime}=P \cup\left\{S^{\prime} \rightarrow \alpha \mid S \rightarrow \alpha \in P\right\}
$$
Allora
\begin{enumerate}
    \item $G^{\prime}$ è una grammatica di tipo $i, i=1,2,3$;
    \item $L\left(G^{\prime}\right)=L(G)$;
    \item Il simbolo iniziale di S' non compare alla destra di nessuna produzione in $G^{\prime}$.
\end{enumerate}

Esempio: Sia G la grammatica definita da
$$
S \rightarrow a S b \mid a b
$$
La grammatica $G^{\prime}$ definita da
$$
S^{\prime} \rightarrow a S b \mid a b, \quad S \rightarrow a S b \| a b
$$
è equivalente a $G$ e inoltre il simbolo iniziale non compare alla destra di nessuna produzione in $G$.

\vspace{5mm}

Sia $S^{\prime}$ il simbolo iniziale di $G^{\prime} .$ Se $S^{\prime}$ non appare alla destra di nessuna produzione in $G^{\prime}$ e aggiungiamo $S^{\prime} \rightarrow \epsilon$, questa produzione potrebbe essere usata solo come come primo (e unico) passo in una derivazione.

\subsubsection{Estensione della definizione}
Una grammatica $G=(V, T, P, S)$, con $V, T$ insiemi finiti e disgiunti, $S \in V$ è
\begin{itemize}
    \item di tipo 1 se ogni produzione in $P$ ha la forma $\alpha \rightarrow \beta$, $\alpha, \beta \in(V \cup T)^{+}$e $|\beta| \geq|\alpha|$.
    \item di tipo 2 se ogni produzione in $P$ ha la forma $A \rightarrow \beta$, con $A \in V$ e $\beta \in(V \cup T)^{+}$
    \item di tipo 3 se ogni produzione in $P$ ha la forma $A \rightarrow a B$ o $A \rightarrow a$, dove $A, B$ sono variabili e a è un terminale.
\end{itemize}
In più permettiamo che $S \rightarrow \epsilon$ sia in $P$ a condizione che $S$ non compaia alla destra di nessuna produzione.

Confrontiamo le classi di linguaggi generati con la vecchia e la nuova definizione.
\begin{itemize}
    \item Se $L$ è generato da una grammatica di tipo $i, i=1,2,3$, secondo la nuova definizione ed $\epsilon \notin L$, allora $L$ è generato da una grammatica di tipo $i, i=1,2,3$, secondo la vecchia definizione.
    \item Se L è generato da una grammatica $G$ di tipo $i, i=1,2,3$, secondo la nuova definizione ed $\epsilon \in L$, elimino in $G$ la produzione $S \rightarrow \epsilon$ (usata solo per generare $\epsilon$ ) e ottengo una grammatica che genera $L \backslash\{\epsilon\}$ ed è di tipo $i, i=1,2,3$, secondo la vecchia definizione.
    \item Se $L$ è generato da una grammatica di tipo $i, i=1,2,3$, secondo la vecchia definizione, allora $L$ è generato da una grammatica di tipo $i, i=1,2,3$, secondo la nuova definizione.
\end{itemize}

Data $G=(V, T, P, S)$ di tipo $i, i \geq 1$, (vecchia definizione) che genera $L(G)$, possiamo trovare una grammatica $G^{\prime}=\left(V^{\prime}, T, P^{\prime}, S^{\prime}\right)$ di tipo $i, i \geq 1$, (nuova definizione) che genera $L(G) \cup\{\epsilon\}$ come segue
\begin{itemize}
    \item introduciamo una nuova variabile $S^{\prime}$ che sarà il simbolo iniziale di $G^{\prime}: V^{\prime}=V \cup\left\{S^{\prime}\right\}$
    \item definiamo $P^{\prime}$ come l'insieme delle produzioni in $P$ più tutte le produzioni $S^{\prime} \rightarrow \alpha$, dove $S \rightarrow \alpha \in P$.
    \item aggiungiamo la produzione $S^{\prime} \rightarrow \epsilon$. Quindi
$$
P^{\prime}=P \cup\left\{S^{\prime} \rightarrow \alpha \mid S \rightarrow \alpha \in P\right\} \cup\left\{S^{\prime} \rightarrow \epsilon\right\}
$$
\end{itemize}

Alla gerarchia di grammatiche introdotta corrisponde una gerarchia sui linguaggi generati.

Un linguaggio $L$ è di tipo i se esiste una grammatica $G$ di tipo i tale che $L=L(G)$.

Otteniamo quattro classi di linguaggi, i linguaggi di tipo 0,1, 2,3 . Le quattro classi formano la
\textbf{Gerarchia di Chomsky}.

\vspace{5mm}

Un linguaggio $L$ è di tipo i se esiste una grammatica $G$ di tipo i tale che $L=L(G)$.

Otteniamo quattro classi di linguaggi, i linguaggi di tipo 0,1, 2,3 .
Che relazione c'è tra le quattro classi?

Siccome la classe delle grammatiche di tipo $i$ include strettamente quella delle grammatiche di tipo $i+1$, possiamo dire che ogni linguaggio di tipo $i+1$ è anche di tipo $i$, $0 \leq i \leq 2$.
Cioè la classe dei linguaggi di tipo $i+1$ è contenuta in quella dei linguaggi di tipo $i, 0 \leq i \leq 2$.
Possiamo però dimostrare il seguente più forte risultato.

\vspace{5mm}

Teorema

La classe dei linguaggi di tipo $i+1$ è contenuta strettamente in quella dei linguaggi di tipo $i, 0 \leq i \leq 2$.

Per provare questo teorema occorre esibire:ù
\begin{itemize}
    \item Un linguaggio generato da una grammatica di tipo zero che
non può essere generato da una grammatica dipendente dal
contesto.
    \item  Un linguaggio generato da una grammatica dipendente dal
contesto che non può essere generato da una grammatica
libera dal contesto.
    \item Un linguaggio generato da una grammatica libera dal contesto
che non può essere generato da una grammatica regolare.
\end{itemize}
Abbiamo definito una gerarchia di grammatiche e una
corrispondente gerarchia di linguaggi.
Vedremo una corrispondente gerarchia di automi che ci
permetterà di fornire gli esempi di linguaggi che separano le
classi.

\subsubsection{Linguaggi formali e di programmazione}
Esaminiamo, attraverso alcuni esempi, le relazioni tra le classi di linguaggi appena introdotte e alcuni aspetti dei linguaggi di programmazione, astraendo da molti dettagli di cui si deve tener conto quando si tratti di linguaggi di programmazione reali.
$$
\begin{aligned}
&L_{1}=\left\{w c w \mid w \in\{a, b\}^{+}\right\} \\
&L_{1}^{\prime}=\left\{a^{n} b^{m} c^{n} d^{m} \mid n, m \geq 1\right\}
\end{aligned}
$$
Questi linguaggi sono di tipo 1 (dipendenti dal contesto), ma non context-free.

Un'interpretazione di $L_{1}$ vede il prefisso $w$ come la dichiarazione di un identificatore con tale nome, c come (una parte di) un programma, e il suffisso $w$ come l'uso dell'identificatore medesimo.
Per quanto riguarda le stringhe di $L_{1}^{\prime}$ potremmo supporre di avere due procedure con $n$ ed $m$ parametri rispettivamente. Allora possiamo interpretare $a^{n}$ e $b^{m}$ come i parametri formali che appaiono nella dichiarazione delle procedure in questione e $c^{n}$ e $d^{m}$ come i parametri attuali delle loro chiamate.

\vspace{5mm}

In generale, un linguaggio di programmazione non è
context-free.

La sintassi dei linguaggi di programmazione è data in termini
di linguaggi context-free, e quindi l'aspetto definizione-uso
contestuale non è esprimibile da una grammatica libera.
Infatti, il controllo che un identificatore sia definito prima del
suo uso o che i parametri attuali siano tanti quanti i formali
viene solitamente fatto, per ragioni di efficienza, da un
componente specifico dei compilatori o degli interpreti.

Come ulteriore esempio, interpretando il carattere a come l'apertura di una delle tante parentesi usate nei programmi, per esempio (, e il carattere $b$ come la parentesi chiusa, per esempio ), una stringa appartiene a $\left\{a^{n} b^{n} \mid n \geq 1\right\}$ se tutte le parentesi aperte vengono chiuse.

Più in generale, una stringa di parentesi (, ) è bilanciata se è
possibile associare una parentesi aperta a quella chiusa
immediatamente alla sua destra e, cancellandole e ripetendo il
procedimento, si giunge a cancellarle tutte.

\vspace{5mm}

Una stringa di parentesi $(,)$ è bilanciata se soddisfa le seguenti due regole.
(1) Una stringa bilanciata contiene lo stesso numero di parentesi aperte e chiuse.
(2) Seguendo la stringa da sinistra a destra il suo profilo non diventa mai negativo, dove il profilo è il valore ottenuto sottraendo al numero delle parentesi aperte incontrate il numero delle parentesi chiuse incontrate.

Ad esempio ()()$,(()()), \epsilon$ sono bilanciate mentre $)($ e $(()$ non lo sono.

\subsubsection{Definizione ricorsiva delle stringhe di parentesi bilanciate}
\begin{itemize}
    \item PASSO BASE: La stringa vuota $\epsilon$ è una stringa di parentesi bilanciata.
    \item PASSO RICORSIVO: Se $x$ e $y$ sono stringhe di parentesi bilanciate, allora anche $(x) y$ è una stringa di parentesi bilanciata.
$G_{b a l}=(\{B\},\{(,)\}, P, B$, dove $P$ consiste nelle produzioni $B \rightarrow B B|(B)| \epsilon$
genera tutte e sole le stringhe di parentesi bilanciate.
\end{itemize}






\let\cleardoublepage\clearpage
\subsubsection{}
% \xRightarrow[G]{*}
% \underset{r m}{\stackrel{*}{\Rightarrow}}