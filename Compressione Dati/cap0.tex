\setcounter{chapter}{-1}
\chapter{Informazioni sul corso}
Il corso è diviso in due parti: nella prima parte si affronteranno le lezioni teoriche, nella seconda parte si organizzerà il progetto. La modalità d'esame prevede appunto delle domande sulla prima parte e l'esposizione del progetto. Il progetto riguarda l'approfondimento su un argomento di compressione dati che si trovano "al confine" tra lo stato dell'arte e la ricerca, nel senso che è possibile ampliare un progetto portato in passato aggiungendo nuovi algoritmi oppure partire ex novo da un argomento nuovo, andando a cercare materiale su articoli, pubblicazioni, ecc.

Un buon progetto è diviso in tre parti:
\begin{itemize}
    \item 1a parte: esposizione di un argomento da approfondire;
    \item 2a parte: soluzioni esistenti in letteratura o sperimentali al problema esposto;
    \item 3a parte: testare e modificare le soluzioni esistenti o trovarne di nuove.
\end{itemize}
I gruppi devono essere di 4 o 5 persone.


\let\cleardoublepage\clearpage