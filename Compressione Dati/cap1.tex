\chapter{Introduzione}
Che cos'è la compressione dati? Non esiste una definizione precisa: intuitivamente è il processo che permette di codificare un insieme di dati \textit{D} in un altro insieme di dati \textit{D\'} che ha dimensioni inferiori. Bisogna però essere in grado di risalire a \textit{D} partendo da \textit{D\'} con un programma che mi permetta di ricostruire l'insieme di dati iniziali esattamente (compressione lossless), o di risalire ad una approssimazione accettabile dell'insieme di dati iniziale (compressione lossy). 

Originariamente la compressione dati veniva utilizzata per minimizzare l'utilizzo dello spazio su disco, in quanto c'erano dischi piccoli che dovevano ospitare grandi basi di dati. La compressione dati però ha un compito fondamentale nella comunicazione: senza di essa oggi non avremmo la televisione in alta definizione, gli smartphone, lo streaming via internet, in quanto tutti i dati che viaggiano in rete sono dati che vengono compressi appunto per aumentare la banda di comunicazione o per inviare più trasmissioni contemporaneamente sullo stesso canale.

\section{Compressione lossless e lossy}
Esistono due tipi di compressione:
\begin{itemize}
    \item \textbf{Compressione lossless (senza perdita)} - viene usata quando i dati devono essere compressi per qualche motivo, ma poi devono essere decompressi senza che venga persa alcuna informazione. Dopo aver decompresso il dato, questo è identico all'originale. È anche detta \textbf{compressione bit preserving} o compressione reversibile. Viene utilizzata su alcuni tipi di dati specifici, come ad esempio dati monodimensionali (programmi oggetto o sorgente, testo, ecc.). La compressione lossless viene utilizzata anche in casi in cui l'acquisizione dei dati è molto costosa e quindi non voglio perdere un singolo bit di informazione (si pensi ai dati raccolti nello spazio e trasmessi sulla Terra); oppure quando non sono ancora note le operazioni da effettuare sui dati raccolti, quindi li si comprime facendo in modo che questi possano essere recuperati interamente. La compressione lossless porta a rapporti di compressione inferiori rispetto alla lossy: file di testo tipicamente vengono compressi con rapporto\footnote{Un rapporto 2 a 1 indica che il file compresso pesa la metà dell'originale, un rapporto 4 a 1 indica che il compresso pesa un quarto dell'originale, e così via.} 8 a 1 o 10 a 1, mentre immagini con rapporto 2 a 1. La compressione lossless è molto importante anche nella trasmissione delle informazioni perché anche solo risparmiando il 10\% della banda porta un risparmio economico.
    \item \textbf{Compressione lossy (con perdita)} - nella compressione lossy non c'è più un unico parametro che ci dice di quanto è stato compresso il file, ma qui si parla di due parametri bilanciati tra di loro: l'ammontare della compressione e la \textbf{fidelity}, ovvero la fedeltà del file decompresso rispetto all'originale. È possibile utilizzare varie combinazioni di questi parametri: si può ottenere una fedeltà massima aumentando il \textit{bit rate} o lo si può diminuire per raggiungere una fedeltà fissata. 
\end{itemize}


\let\cleardoublepage\clearpage