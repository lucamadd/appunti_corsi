\chapter{I reati informatici nell'ordinamento giuridico italiano}

Negli anni la legislatura si è dovuta adeguare a nuove ipotesi di reato a causa delle tecnologie sempre emergenti. Esistono infatti dei reati che si commettono soltanto sulla rete e che costituiscono una novità per come vengono commessi.  
 
\vspace{5mm}
 
\textbf{Art. 1 lett.a) Definizione di sistema informatico} -  \textit{"qualsiasi apparecchiatura o gruppi di apparecchiature interconnesse o collegate, una o più delle quali, in base ad un programma, compiono l'elaborazione automaica dei dati..."} (Convenzione di Budapest 2001)
 
\vspace{5mm}
 
Il primo problema che si è posto è stato definire l'ambito di applicazione della legislatura in ambito informatico, in quanto un attacco può rappresentare diversi reati: un furto? Una truffa? Non essendo attacchi materiali, occorre stabilire definizioni chiare per muoversi bene in quest'ambito. 

\vspace{5mm}

\textbf{Art. 1 lett.b) Definizione di dati informatici} - \textit{"qualunque presentazioni di fatti, informazioni o concetti in forma suscettibile di essere utilizzata in un sistema computerizzato, incluso un programma in grado di consentire ad un sistema computerizzato di svolgere una funzione"} (Convenzione di Budapest 2001)

\vspace{5mm}

Vediamo alcune delle leggi che rappresentano la normativa reati informatici:

\vspace{5mm}

\textbf{L. 3.5.1991 n. 143} conv. in L. 5.7.1991 n. 197 (art. 12) - \textit{"indebito utilizzo di carte di credito, di pagamento e di documenti che abilitano al prelievo di denaro"}

\vspace{5mm}

\textbf{D.Lgs 30.6.2003 n.196} - \textit{"protezione dei dati personali"}

\vspace{5mm}

\textbf{D.Lgs 9.4.2003 n.68} - \textit{"diritto di autore"} (L. n. 633/1941)

\vspace{5mm}

\textbf{D. Lgs 70/2003} - attuazione della direttiva 31/2000 CE sul commercio elettronico (artt. 16 e 17) e responsabilità, penale e civile, dei provider

\vspace{5mm}

\textbf{D. Lgs N.231/2001} sulla responsabilità degli enti, società ed associazione anche prive di personalità giuridica

\vspace{5mm}

\textbf{L. n. 155/2005} - contrasto al terrorismo

\vspace{5mm}

\textbf{L. n. 38/2006} - contrasto sfruttamento sessuale dei bambini e la pedopornografia anche a mezzo Internet

\vspace{5mm}

\textbf{L. n. 231/2007} - contrasto ad attività di riciclaggio

\vspace{5mm}

\textbf{L. n. 547/1993} - \textit{"modificazioni alle norme del codice penale e del codice di procedura penale in tema di criminalità informatica"}

\vspace{5mm}

Una legge importantissima da ricordare è la seguente, discussa alla convenzione di Budapest il 23 novembre 2001 e convertita in legge (in Italia) nel 2008:

\vspace{5mm}

\textbf{L. n. 48/2008} - ratifica ed esecuzione Convenzione Consiglio d'Europa (Budapest 2001) sulla criminalità informatica e norme di adeguamento dell'ordinamento interno

\vspace{5mm}

Con la convenzione di Budapest si è cercato di creare una normativa che andasse bene per tutti a livello europeo, in quanto ogni paese, in base a ciò che ritiene opportuno, può convertire in legge le normative discusse in questa convenzione.

A seguito del 2008 sono nate norme e leggi che poi sono diventate punto di riferimento per i tecnici e per le forze dell'ordine che per fare ispezione di dati, hanno potuto fare riferimento alle norme introdotte da queste leggi:

\vspace{5mm}

\textbf{L. n. 38/2009} in maniera di sicurezza pubblica e di contrasto alla violenza sessuale, nonché in tema di atti persecutori (\textit{stalking})

\vspace{5mm}

\textbf{L. n.172/2012} - attuazione della Convenzione del Consiglio d'Europa (Lanzarote 2007) in materia di protezione dei minori dallo sfruttamento e dagli abusi sessuali



\section{Reati informatici in senso lato}
Distinguiamo i reati informatici in due categorie: la prima è rappresentata dai reati informatici in senso lato, ovvero quelli che già esistevano e che sono stati rivisti dal legislatore perché sono ipotesi di reato che possono essere commesse anche grazie all'ausilio di strumenti elettronici. Comprendono:
\begin{itemize}
    \item \textbf{diffamazione} (art. 595 c.p.) - via internet, Social Network, forum, ecc.
    \item \textbf{ingiuria} (art. 594 c.p.) - via email, sms, chat privata\footnote{Ingiuria e diffamazione sono molto simili: in entrambe si va a ledere la reputazione di una persona, ma la diffamazione presuppone che vengono messe a conoscenza dell'offesa almeno due persone.}
    \item \textbf{favoreggiamento personale} (art. 378 c.p.) - es. cancellazione della memoria di un computer di tracce di un reato commesso da altri
    \item \textbf{associazione a delinquere} (art. 416 c.p.) - finalizzata alla commissione di reati informatici
    \item \textbf{falsificazione di un documento} (artt. 476 e 491 bis c.p.) - con mezzi informatici
    \item \textbf{violazione dei diritti d'autore} (art. 171 bis L. 633/1941) - in particolare la duplicazione abusiva di software. Quest'ultima è abbonata dalla Corte di Cassazione se per uso personale
    \item \textbf{truffa} (art. 640 c.p.) - che può essere commessa ad esempio vendendo un oggetto che non si possiede
    \item \textbf{estorsione} (art. 629 c.p.)
    \item \textbf{ricettazione} (art. 648 c.p.)
    \item \textbf{riciclaggio} 8art. 648 bis c.p.)
    \item \textbf{sostituzione di persona} (art. 494 c.p.) - nei casi di account falsi che non costituiscono un fenomeno mediatico non rappresentato un reato.
    \item \textbf{stalking} (art. 612 bis c.p.) - anni fa fu introdotta la legge Carfagna nel pacchetto sicurezza, che però a differenza di altri paesi, non ha introdotto il cyber stalking. Tuttavia si è fatto in modo che anche lo stalking che avviene telematicamente può rappresentare ipotesi di reato, e in alcuni casi può essere anche peggiore, in quanto non sempre è possibile risalire all'identità del mittente dello stalking.
    \item \textbf{esercizio arbitrario delle proprie ragioni con violenza sulle cose} (art. 392 c.p.) - può interessare i tecnici perché è l'esempio del sistemista o dello sviluppatore che dopo aver creato il software che viene poi utilizzato dall'azienda, e viene licenziato, limita l'utilizzo del sofware o lo inibisce in quanto solo lui conosce i meccanismi interni.
    \item \textbf{attentato a impianti di pubblica utilità} (art. 420 c.p.)
    \item \textbf{Pornografia minorile} (art. 600-ter c.p.) - pene (da 6 a 12 anni di reclusione) per chi utilizza minori, realizza, produce, induce minori a partecipare ad esibizioni pedopornografiche; commercia, distribuisce, divulga, pubblicizza; offre, cede (anche gratuitamente) materiale pedopornografico.  
    \item \textbf{Detenzione di materiale pornografico} (art. 600-quater 1 c.p.) - pene fino a 3 anni di reclusione, sempre per quanto riguarda minori, nei casi in cui il detentore del materiale sia consapevole di possederlo (non si applica nel caso di ricezione fortuita di materiale pedopornografico che viene subito cestinato).
    \item \textbf{Pornografia virtuale} (art. 600-quater 1 c.p.) - rappresenta ipotesi di reato anche il disegno di un minore in atti sessuali
    \item \textbf{Istigazione a pratiche di pedofilia e di pedopornografia (art. 414 bis c.p.)} - introdotto dalla L. n. 172/2012 - \textit{"Salvo che il fatto costituisca più grave reato, chiunque, con qualsiasi mezzo e con qualsiasi forma di espressione, pubblicamente istiga a commettere, in danno di minorenni, uno o più delitti previsti dagli articoli 600-bis, 600-ter e 600-quater, anche se relativi al materiale pornografico di cui all'art. 600-quater, 600-quinquies, 609-bis, 609-quater e 609-quinquies è punito con la reclusione da un anno e sei mesi a cinque anni. Alla stessa pena soggiace anche chi pubblicamente fa l'apologia di uno o più delitti previsti dal primo comma. Non possono essere invocate, a propria scusa, ragioni o finalità di carattere artistico, letterario, storico o di costume."} 
    \item \textbf{Adescamento di minorenni} (art. 609-undecies c.p.) - anch'esso introdotto dalla L. n. 172/2012 - \textit{"Chiunque, allo scopo di commettere i reati di cui agli articoli 600, 600 bis, 600 ter e 600 quater, anche se relativi al materiale pornografico di cui all'articolo 600 quater, 600 quinquies, 609 bis, 609 quater, 609 quinquies e 609 octies, adesca un minore di anni sedici, è punito, se il fatto non costituisce più grave reato, con la reclusione da uno a tre anni. Per adescamento si intende qualsiasi atto volto a carpire la fiducia del minore attraverso artifici, lusinghe o minacce posti in essere anche mediante l'utilizzo della rete internet o di altre reti o mezzi di comunicazione"}. L'adescamento costituisce reato a prescindere dall'atto sessuale conseguente.
    \item \textbf{Revenge Porn} (art. 612 ter c.p. - diffusione illecita di immagini o video sessualmente espliciti) - introdotto con L. n.69 del 19 luglio 2019 - \textit{"Salvo che il fatto costituisca più grave reato, chiunque, dopo averli realizzati o sottratti, invia, consegna, cede, pubblica o diffonde immagini o video di organi sessuali o a contenuto sessualmente esplicito, destinati a rimanere privati, senza il consenso delle persone rappresentate, è punito con la reclusione da uno a sei anni e una multa da 5.000 a 15.000 euro"}. La stessa pena si applica a chi, avendo ricevuto o comunque acquisito le immagini o i video li invia, consegna, cede, pubblica o diffonde senza il consenso delle persone rappresentate al fine di recare loro nocumento. La pena è aumentata se i fatti sono commessi dal coniuge, anche separato o divorziato, o da persone che è o è stata legata da relazione affettiva alla persona offesa ovvero se i fatti sono commessi attraverso strumenti informatici o telematici
\end{itemize}

\section{Reati informatici in senso stretto}
Questo tipo di reati possono essere commessi solo con l'ausilio di mezzi informatici.

\vspace{5mm}

\textbf{Documenti informatici} (art. 491 bis c.p.) - \textit{"Se alcuna delle falsità previste dal seguente capo riguarda un documento informatico pubblico avente efficacia probatoria, si applicano le disposizioni del capo stesso concernenti gli atti pubblici"} - riguarda la falsificazione di un documento informatico

\vspace{5mm}

La definizione di documento informatico è \textit{"qualsiasi documento conservato in forma elettronica, in particolare testo o registrazione sonora, visiva o audiovisiva"} (Regolamento EIDAS 910/2014 UE)

\vspace{5mm}


\textbf{Falsa dichiarazione o attestazione al certificatore di firma elettronica sull'identità o su qualità personali proprie o di altri} (art. 495-bis c.p.) - a un documento elettronico non sono negati gli effetti giuridici e l'ammissibilità come prova in procedimenti giudiziali per il solo motivo della sua forma elettronica (Definizione art. 46 Regolamento EIDAS 910/2014 UE)

\vspace{5mm}

\textbf{Accesso abusivo a un sistema informatico o telematico} (art. 615-ter c.p.) - \textit{"Chiunque abusivamente si introduce in un sistema informatico o telematico protetto da misure di sicurezza ovvero vi si mantiene contro la volontà espressa o tacita di chi ha il diritto di escluderlo, è punito con la reclusione fino a tre anni"} - quest'articolo segue l'articolo sulla violazione di domicilio, tuttavia nel caso di accesso abusivo a un sistema informatico è specificato che il sistema deve possedere misure di sicurezza.

\vspace{5mm}

\textbf{Detenzione e diffusione abusiva di codici di accesso a sistemi informatici o telematici} (art. 615-quater c.p.) - \textit{"Chiunque, al fine di procurare a sé o ad altri un profitto o di arrecare ad altri un danno, abusivamente si procura, riproduce, diffonde, comunica o consegna codici, parole chiave, o altri mezzi idonei all'accesso ad un sistema informatico o telematico, protetto da misure di sicurezza, o comunque fornisce indicazioni o istruzioni idonee al predetto scopo è punito con la reclusione sino ad un anno e con la multa sino a euro 5.164"} - 

\vspace{5mm}

\textbf{Diffusione di apparecchiature, dispositivi, o programmi informatici diretti a danneggiare o interrompere un sistema informatico o telematico} (art. 615-quinquies c.p.)  \textit{"Chiunque, allo scopo di danneggiare illecitamente un sistema informatico o telematico, le informazioni, i dati o i programmi in esso contenuti o ad esso pertinenti ovvero di favorire l'interruzione, totale o parziale, o l'alterazione del suo funzionamento, si procura, produce, riproduce, importa, diffonde, comunica, consegna o, comunque, mette a disposizione di altri apparecchiature, dispositivi o programmi informatici, è punito con la reclusione fino a due anni e con la multa sino a euro 10.329"}

\vspace{5mm}

\textbf{Intercettazione, impedimento o interruzione illecita di comunicazioni informatiche o telematiche} (art. 617-quater c.p.)

\vspace{5mm}

\textbf{Installazione di apparecchiature atte a intercettare, impedire o interrompere comunicazioni informatiche o telematiche} (art. 617-quinquies c.p.)

\vspace{5mm}

\textbf{Falsificazione, alterazioni o soppressione del contenuto di comunicazioni informatiche o telematiche} (art. 617-sexies c.p.)

\vspace{5mm}

\textbf{Altre comunicazioni e conversazioni} (art. 623-bis c.p.)

\vspace{5mm}

\textbf{Danneggiamento di informazioni, dati e programmi informatici} (art. 635-bis c.p.) - \textit{"Salvo che il fatto costituisca più grave reato, chiunque distrugge, deteriora, cancella, altera o sopprime informazioni, dati o programmi informatici altrui è punito, a querela della persona offesa, con la reclusione da sei mesi a tre anni. Se il fatto è commesso con violenza alla persona o con minaccia ovvero con abuso della qualità di operatore del sistema, la pena è della reclusione da uno a quattro anni"}

\vspace{5mm}

\textbf{Danneggiamento di informazioni, dati e programmi informatici utilizzati dallo Stato o da altro ente pubblico o comunque di pubblica utilità} (art. 635-ter c.p.)

\vspace{5mm}

\textbf{Danneggiamento di sistemi informatici o telematici} (art. 635-quater c.p.) - \textit{"Salvo che il fatto costituisca più grave reato, chiunque distrugge, deteriora, cancella, altera o sopprime informazioni, dati o programmi informatici altrui è punito, a querela della persona offesa, con la reclusione da sei mesi a tre anni. Se il fatto è commesso con violenza alla persona o con minaccia ovvero con abuso della qualità di operatore del sistema, la pena è della reclusione da uno a quattro anni"}

\vspace{5mm}

\textbf{Danneggiamento di sistemi informatici o telematici di pubblica utilità} (art. 635-quinquies c.p.)

\vspace{5mm}

\textbf{Frode informatica} (art. 640-ter c.p.) - \textit{"Chiunque, alterando in qualsiasi modo il funzionamento di un sistema informatico o telematico o intervenendo senza diritto con qualsiasi modalità su dati, informazioni o programmi contenuti in un sistema informatico o telematico o ad esso pertinenti, procura a sé o ad altrui un ingiusto profitto con altrui danno, è punito con la reclusione da sei mesi a tre anni e con la multa da 51 euro a 1032 euro. La pena è della reclusione da uno a cinque anni e della multa da 309 euro a 1549 euro se ricorre una delle circostanze previste dal numero 1) del secondo comma dell'articolo 640, ovvero se il fatto è commesso con abuso della qualità di operatore del sistema. La pena è della reclusione da due a sei anni e delle multa da euro 600 a euro 3000 se il fatto è commesso con furto o indebito utilizzo dell'identità digitale in danno di uno o più soggetti"}

\vspace{5mm}

\textbf{Frode informatica del soggetto che presta servizi di certificazione di firma elettronica} (art. 640-quinquies c.p.)

\vspace{5mm}

\section{Altre informazioni sulla computer forensics}
Negli ultimi anni Internet sta incoraggiando un comportamento antinormativo, aggressivo e disinibito, perché dietro lo schermo di un computer si riescono a compiere più facilmente illeciti: le conseguenze sono l'aumento del numero dei reati, soprattutto quelli sulla rete, e dunque è aumentata la necessità di acquisire prove digitali da usare nei processi. C'è bisogno quindi di attuare una serie di meccanismi per salvaguardare queste prove, che sono necessarie nei processi di reati informatici e nei processi di reati comuni, vedi ad esempio un qualsiasi contratto \textit{point and click}, quando acquistiamo un biglietto aereo o facciamo una qualsiasi prenotazione.

\vspace{5mm}

\textbf{Nomina consulenti tecnici} (art. 359 c.p.p.) - Il pubblico ministero, quando procede ad accertamenti, rilievi segnaletici, descrittivi o fotografici e ad ogni altra operazioni tecnica per cui sono necessarie specifiche competenze, può nominare e avvalersi di consulenti, che non possono rifiutare la loro opera. - Da parecchi anni quando c'è necessità di ispezionare vari computer, ci si rivolge a varie persone.

\vspace{5mm}

\textbf{Nomina del perito} (art. 221 c.p.p) - \textit{"Il giudice nomina il perito scegliendolo tra gli iscritti negli appositi albi o tra persone fornite di particolare competenza nella specifica disciplina"}.

\vspace{5mm}

\textbf{Attività investigativa del difensore} (art. 327-bis c.p.p)

\vspace{5mm}

\textbf{Investigazioni difensive} (art. 391 bis - 391 decies c.p.p.)