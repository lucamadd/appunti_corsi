\subsubsection{DOMANDA 1}
Descrivere il significato del seguente script BitCoin:

\begin{center}
    \texttt{HASH160 9af61346ce0aa2dffcf697352b4b704c84dcbaff EQUAL}
\end{center}

\subsubsection{DOMANDA 2}
Descrivere cosa realizza il seguente smart contract in Solidity:
\begin{lstlisting}[language=Solidity]
pragma solidity >=0.4.24 <0.6.0;

contract ReceiverPays {
    address owner = msg.sender;

    mapping(uint256 => bool) usedNonces;

    constructor() public payable {}

    function claimPayment(uint256 amount, uint256 nonce, bytes memory signature) public {
        require(!usedNonces[nonce]);
        usedNonces[nonce] = true;

        bytes32 message = prefixed(keccak256(abi.encodePacked(msg.sender, amount, nonce, this)));

        require(recoverSigner(message, signature) == owner);

        msg.sender.transfer(amount);
    }

    function kill() public {
        require(msg.sender == owner);
        selfdestruct(msg.sender);
    }

    function splitSignature(bytes memory sig)
        internal
        pure
        returns (uint8 v, bytes32 r, bytes32 s)
    {
        require(sig.length == 65);

        r := mload(add(sig, 32))
        s := mload(add(sig, 64))
        v := byte(0, mload(add(sig, 96)))
        
        return (v, r, s);
    }

    function recoverSigner(bytes32 message, bytes memory sig)
        internal
        pure
        returns (address)
    {
        (uint8 v, bytes32 r, bytes32 s) = splitSignature(sig);

        return ecrecover(message, v, r, s);
    }

    function prefixed(bytes32 hash) internal pure returns (bytes32) {
        return keccak256(abi.encodePacked("\x19Ethereum Signed Message:\n32", hash));
    }
}

\end{lstlisting}

\subsubsection{DOMANDA 3}
Descrivere cosa realizza il seguente chaincode in Go:
\begin{lstlisting}[language=Go]
package main

import (
    "fmt"

    "github.com/hyperledger/fabric/core/chaincode/shim"
    "github.com/hyperledger/fabric/protos/peer"
)

type SampleChaincode struct {

}

func (t *SampleChaincode) Init(stub shim.ChaincodeStubInterface) peer.Response {
    args := stub.GetStringArgs()
    if len(args) != 2 {
        return shim.Error("Incorrect arguments. Expecting a key and a value")
    }

    err := stub.PutState(args[0], []byte(args[1]))
    if err != nil {
        return shim.Error(fmt.Sprintf("Failed to create asset: %s", args[0]))
    }
    return shim.Success(nil)
}

func (t *SampleChaincode) Invoke(stub shim.ChaincodeStubInterface) peer.Response {
    fn, args := stub.GetFunctionAndParameters()

    var result string
    var err error
    if fn == "set" {
        result, err = set(stub, args)
    } else {
        result, err = get(stub, args)
    }
    if err != nil {
        return shim.Error(err.Error())
    }

    return shim.Success([]byte(result))
}

func set(stub shim.ChaincodeStubInterface, args []string) (string, error) {
    if len(args) != 2 {
        return "", fmt.Errorf("Incorrect arguments. Expecting a key and a value")
    }

    err := stub.PutState(args[0], []byte(args[1]))
    if err != nil {
        return "", fmt.Errorf("Failed to set asset: %s", args[0])
    }
    return args[1], nil
}

func get(stub shim.ChaincodeStubInterface, args []string) (string, error) {
    if len(args) != 1 {
        return "", fmt.Errorf("Incorrect arguments. Expecting a key")
    }

    value, err := stub.GetState(args[0])
    if err != nil {
        return "", fmt.Errorf("Failed to get asset: %s with error: %s", args[0], err)
    }
    if value == nil {
        return "", fmt.Errorf("Asset not found: %s", args[0])
    }
    return string(value), nil
}

func main() {
    err := shim.Start(new(SampleChaincode))
    if err != nil {
        fmt.Println("Could not start SampleChaincode")
    } else {
        fmt.Println("SampleChaincode successfully started")
    }
}

\end{lstlisting}

\subsubsection{DOMANDA 4}
Quale vulnerabilità è presente in questo smart contract in Solidity? Descrivere una possibile soluzione:
\begin{lstlisting}[language=Solidity]
// library contract
contract FibonacciLib {
    uint public start;
    uint public calculatedFibNumber;
    function setStart(uint _start) public {
        start = _start;
    }
    function setFibonacci(uint n) public {
        calculatedFibNumber = fibonacci(n);
    }
    function fibonacci(uint n) internal returns (uint) {
        if (n == 0) return start;
        else if (n == 1) return start + 1;
        else return fibonacci(n - 1) + fibonacci(n - 2);
    }
}

contract FibonacciBalance {
    address public fibonacciLibrary;
    uint public calculatedFibNumber;
    uint public start = 3;    
    uint public withdrawalCounter;
    bytes4 constant fibSig = bytes4(sha3("setFibonacci(uint256)"));
    
    constructor(address _fibonacciLibrary) public payable {
        fibonacciLibrary = _fibonacciLibrary;
    }
    function withdraw() {
        withdrawalCounter += 1;
        require(fibonacciLibrary.delegatecall(fibSig, withdrawalCounter));
        msg.sender.transfer(calculatedFibNumber * 1 ether);
    }
    
    function() public {
        require(fibonacciLibrary.delegatecall(msg.data));
    }
}

\end{lstlisting}
\clearpage
\subsubsection{DOMANDA 5}
Descrivere il seguente chaincode go:
\begin{lstlisting}
package main

import (  
    "fmt"
    "time"
)

func numbers(c1 chan int, done chan bool) {  
    for i := 1; i <= 5; i++ {
        time.Sleep(20 * time.Millisecond)
        c1 <- i
    }
    done <- true
}
func alphabets(c2 chan string, done chan bool) {  
    for i := 'a'; i <= 'e'; i++ {
        time.Sleep(40 * time.Millisecond)
        c2 <- string(i)
    }
    done <- true
}
func main() {  
    c1 := make(chan int, 2)
    c2 := make(chan string, 2)
    done := make(chan bool)
    go numbers(c1, done)
    go alphabets(c2, done)
    go func() {
        for i := 1; i <= 5; i++ {
            fmt.Println(<-c1)
        }
    }()
    go func() {
        for i := 1; i <= 5; i++ {
            fmt.Println(<-c2)
        }
    }()
    
    for i := 0; i < 2; i++ {
        <-done
    }
    fmt.Println("main terminated")
}

\end{lstlisting}