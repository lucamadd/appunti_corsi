\chapter{Stringhe e linguaggi}

\section{Alfabeto}
Un alfabeto è un insieme finito di elementi (chiamati lettere
o simboli o caratteri (terminali))

\vspace{5mm}

\textbf{Esempio:} L’alfabeto delle lettere romane minuscole è
\(\Sigma = {a, b, c, \dots, z}\) 

\vspace{5mm}

\textbf{Esempio:} L’alfabeto delle cifre arabe è
\(\Sigma = {0, 1, \dots , 9}\) 

\vspace{5mm}

\textbf{Esempio:} L’alfabeto binario è
\(\Sigma = {0, 1}\) 

\vspace{5mm}

Sia \(\Sigma = {a_1, \dots, a_k}\) un alfabeto di \(k\) simboli. La cardinalità di \(\Sigma\) è \(k\). In simboli: \(|\Sigma| = k\).

Una stringa (o parola) su un alfabeto è una sequenza (finita)
di simboli dell’alfabeto.
Ossia è un insieme ordinato con eventuali ripetizioni di
caratteri.

\vspace{5mm}

\textbf{Esempio:} Sia \(\Sigma = {a, b}\).
\(aaba, aaa, abaa, b\) sono stringhe.
\vspace{5mm}

\textbf{Esempio:} 0131 `e una stringa sull’alfabeto \(\Sigma = {0, 1, 2, \dots, 9}\).
\vspace{5mm}

\textbf{Esempio:} 0101 `e una stringa sull’alfabeto \(\Sigma = {0, 1}\).

La cardinalit`a di un linguaggio (finito) `e il numero delle sue
stringhe.
\vspace{5mm}

\textbf{Esempio:}
\(|L2| = |{aba, aab}| = 2\)
Un linguaggio finito `e un linguaggio che ha un numero finito
di stringhe.
Un linguaggio infinito `e un linguaggio non finito.
Un esempio di linguaggio finito: il linguaggio vuoto \(\emptyset\).
\(|\emptyset| = 0\).

Nota: non solo linguaggi finiti.
Infatti i linguaggi finiti non sono di solito interessanti.
Tutti i nostri alfabeti sono finiti, ma la maggior parte dei
linguaggi che incontreremo sono infiniti.

\section{Stringhe}
Il numero di occorrenze di un carattere a in una stringa \(x\)
viene indicato da \(|x|_a\).
\vspace{5mm}

\textbf{Esempio:}
\[|aab|_a = 2, |baa|_a = 2, |baa|_b = 1, |baa|_c = 0\]
La lunghezza di una stringa \(x\) `e il numero di simboli in \(x\).
La lunghezza di \(x\) `e denotata con \(|x|\).
\textbf{Esempio}: \(|ab| = 2, |abaa| = 4\).

Due stringhe
\[x = a_{1}a_{2} \dots a_{h}, y = b_{1}b_{2} \dots b_{k} ,\]
con \(a_{i}, b_{j} \in \Sigma, 1 \leq i \leq h, 1 \leq j \leq k\), si dicono eguali se \(h = k e a_i = b_i\)
, per \(i = 1, \dots , h\).
In due stringhe uguali i caratteri letti ordinatamente da
sinistra a destra coincidono.
Esempio: \(aba \neq baa, baa \neq ba\).

\subsection{Operazioni sulle stringhe}
Date le stringhe
\[x = a_{1}a_{2} \dots a_{h}, y = b_{1}b_{2} \dots b_k ,\]
con \(a_i, b_j \in \Sigma, 1 \leq i \leq h, 1 \leq j \leq k\), la concatenazione (di
\(x\) e \(y\)) `e definita da
\[x \dot y = a_{1}a_{2} \dots a_{h}b_{1}b_{2} \dots b_{k}\]
La concatenazione di due stringhe \(x\) e \(y\) `e spesso denotata \(xy\)
(invece che \(x \dot y\)).
\vspace{5mm}
\textbf{Esempio:} \(x =\) vice, \(y =\) capo, \(z =\) stazione \(xy =\) vicecapo,
\(yx =\) capovice \(\neq xy\)
\((xy)z =\) vicecapostazione \(= x(yz)\)

La concatenazione non `e commutativa, in generale \(xy \neq  yx\).
La concatenazione `e associativa:
\[(xy)z = x(yz)\]
(possiamo scrivere senza parentesi la concatenazione di tre o
pi`u stringhe).
\(|xy| = |x| + |y|\)

La stringa vuota \(\epsilon\) `e la stringa che non contiene nessun
simbolo.
Propriet`a della stringa vuota:
\[x \epsilon = \epsilon x = x\]
\[|\epsilon| = 0\]
Nota :
\[\emptyset \neq \epsilon, \emptyset \neq \{\epsilon\}\]

\textbf{Definizione.} Data una stringa \(x\), una sottostringa di \(x\) `e una qualsiasi
sequenza di simboli consecutivi della stringa \(x\). Un prefisso di
\(x\) `e una qualsiasi sequenza di simboli consecutivi iniziali della
stringa \(x\). Un suffisso di \(x\) `e una qualsiasi sequenza di simboli
consecutivi terminali della stringa \(x\).
Se \(x = uyv\) `e la concatenazione di stringhe \(u\), \(y\), \(v\)
(eventualmente vuote) allora:
\begin{itemize}
    \item \(y\) `e una \textbf{sottostringa} di \(x\),
    \item \(u\) `e un \textbf{prefisso} di \(x\),
    \item \(v\) `e un \textbf{suffisso} di \(x\)
\end{itemize}
Una sottostringa (prefisso, suffisso) di \(x\) `e \textbf{propria} se non
coincide con \(x\).

Esempio: La stringa 472 ha
\begin{itemize}
    \item prefissi: \(epsilon\), 4, 47, 472,
    \item suffissi: \(epsilon\), 2, 72, 472,
    \item sottostringhe: \(epsilon\), 4, 7, 2, 47, 72, 472
    \item La stringa 42 non `e sottostringa di 472.
\end{itemize}

L’inversa (o reverse o riflessione) \(w^R\) di una stringa \(w\) `e la stringa
ottenuta scrivendo i caratteri di \(w\) da destra verso sinistra.
\(\epsilon^R = \epsilon\) e se \(w = a_1 \dots a_n\), con \(a_j\)
lettere, allora
\(w^R = a_n, a_{n-1} \dots a_1\).
\textbf{Esempio:} x = roma, \(x^R\) = amor.
Propriet`a:
\[(x^R)^R = x, (xy)^R = y^Rx^R\]

\begin{itemize}
    \item PASSO BASE: \(\epsilon^R = \epsilon\).
    \item PASSO RICORSIVO: Per ogni \(x \in \Sigma*\) e \(\sigma \in \Sigma\), \((x\sigma)^R = \sigma x^R\).
\end{itemize}

Sia \(m \geq 1\) un intero. La potenza m-esima di una stringa \(x\) `e
la concatenazione di \(x\) con s´e stessa \(m - 1\) volte.
Per convenzione la potenza 0 di una stringa `e la stringa vuota.

\vspace{5mm}

\textbf{Definizione}: Sia \(x\) una stringa. Poniamo:
\begin{itemize}
    \item PASSO BASE: \(x^0 = \epsilon\)
    \item PASSO RICORSIVO: \(x^m = x^{m-1}x\), per \(m > 0\).
\end{itemize}
\textbf{Esempi}:
\begin{itemize}
    \item \(x = ab\)
    \item \(x^0 = \epsilon\)
    \item \(x^1 = x = ab\)
    \item \(x^2 = (ab)^2 = abab\)
    \item \(y = a^2 = aa\)
    \item \(y^3 = a^{2}a^{2}a^{2} = a^6\)
    \item \(\epsilon^0 = \epsilon\)
    \item \(\epsilon^2 = \epsilon\)
\end{itemize}
Nota. E necessario racchiudere tra parentesi la stringa da `
elevare alla potenza se ha lunghezza maggiore di uno.
\[(ab)^2 = abab \neq abb = ab^2\]
L’elevamento a potenza ha precedenza rispetto alla
concatenazione.
Anche la riflessione ha precedenza rispetto alla
concatenazione.
\(b^R = b\), quindi \(ab^R = ab\), mentre \((ab)^R = ba \neq ab^R = ab\)

\section{Linguaggi}
Data un’operazione su una stringa, essa si pu`o estendere a
tutte le stringhe di un linguaggio.
Otteniamo cos`ı alcune operazioni sui linguaggi.

La riflessione di un linguaggio L:
\[L^R = \{x | x = y^R \wedge y \in L\}\]
L’insieme dei prefissi propri di un linguaggio L:
\begin{center}
Prefissi(L) = \(\{y | x = yz \wedge x \in L \wedge z \neq \epsilon\}\)
\end{center}

Un linguaggio `e prefisso se non contiene nessuno dei suoi
prefissi propri.
Un linguaggio L `e prefisso se e solo se
\(L \cap\) Prefissi(L) \(= \emptyset\)
Importanza pratica: se nella trasmissione dell’informazione
una parte finale della stringa viene accidentalmente troncata,
l’errore viene individuato.
Nella codifica dell’informazione la decodifica `e immediata.

\textbf{Esempio.} \(L_1 = \{a^{n}b^{n}| n \geq 1\}\) `e prefisso perch´e
Prefissi\((L_1) = \{a^{n}b^{m} | n > m \geq 1\} \cup \{a^{n}| n \geq 0\}\).

\vspace{5mm}

\textbf{Esempio.} \(L_2 = {a^{m}b^{n}| m \geq n \geq 1}\) non `e prefisso perch´e,ad esempio, \(a^{3}b^{2}\) e \(a^{3}b\) sono entrambi in \(L2\).

\vspace{5mm}

\textbf{Esempio.} Ogni linguaggio L che contiene la stringa vuota \(\epsilon\) e
tale che \(L \neq {\epsilon}\) non `e prefisso.

Anche le operazioni su due stringhe possono essere estese a
due linguaggi.

\subsection{Operazioni sui linguaggi}
Anche le operazioni su due stringhe possono essere estese a
due linguaggi.
\subsubsection{Prodotto di linguaggi}
Dati due linguaggi $L^{\prime}$ ed $L^{\prime \prime}$ sull'alfabeto $\Sigma$, il prodotto (o concatenazione) di $L^{\prime}$ ed $L^{\prime \prime}$ è
$$
L^{\prime} L^{\prime \prime}=L^{\prime} \circ L^{\prime \prime}=\left\{x y \mid x \in L^{\prime} \wedge y \in L^{\prime \prime}\right\}
$$
$L^{\prime} L^{\prime \prime}$ è l'insieme di tutte le stringhe che sono concatenazione di una stringa in $L^{\prime}$ e di una stringa in $L^{\prime \prime}$.
Esempio. Siano
$$
\begin{gathered}
L_{1}=\left\{a^{i} \mid i \geq 0, \text { pari }\right\} \\
L_{2}=\left\{b^{j} a \mid j \geq 1, \text { dispari }\right\}
\end{gathered}
$$
risulta
$$
L_{1} L_{2}=\left\{a^{i} b^{j} a \mid(i \geq 0, \text { pari }) \wedge(j \geq 1, \text { dispari })\right\}
$$
Esempi di stringhe in $L_{1} L_{2}$ :
$$
b a, a^{2} b a, a^{4} b a, b^{3} a, a^{2} b^{3} a, a^{4} b^{3} a
$$
\subsubsection{Potenza di un linguaggio}
Sia L un linguaggio sull'alfabeto $\Sigma$. Definiamo:
$$
\begin{aligned}
L^{0} &=\{\epsilon\}, \\
L^{k} &=L^{k-1} L, \quad k \geq 1
\end{aligned}
$$
Nota.
$$
\begin{aligned}
&L^{1}=L \\
&L^{k}=\left\{w_{1} w_{2} \ldots w_{k} \mid w_{i} \in L, 1 \leq i \leq k\right\}, k \geq 0
\end{aligned}
$$
\textbf{Casi particolari:} 
$$
\emptyset^{0} =\{\epsilon\} 
$$
$$
L \cdot \emptyset =\emptyset \cdot L=\emptyset 
$$
$$
L \cdot\{\epsilon\} =\{\epsilon\} \cdot L=L
$$

Nota. Dato un linguaggio $L$ e un intero $m \geq 2$ consideriamo
- il linguaggio $\left\{x^{m} \mid x \in L\right\}$ che ha come elementi le potenze $m$-esime degli elementi di $L$,
- la potenza $m$-esima $L^{m}$ di $L$.
$$
\left\{x^{m} \mid x \in L\right\} \subseteq L^{m}
$$
In generale il primo è incluso ma è diverso dal secondo.
Esempio
$$
L=\{a, b\}, \quad\left\{x^{2} \mid x \in L\right\}=\{a a, b b\}, \quad L^{2}=\{a a, b b, a b, b a\}
$$
Nota. Dato un linguaggio $L$ e un intero $m \geq 2$, in generale $\left\{x^{m} \mid x \in L\right\}$ è incluso strettamente in $L^{m}$.
Cosa accade per $m=0$ ?
Cosa accade per $m=1$ ?
Esistono linguaggi $L$ tali che $\left\{x^{m} \mid x \in L\right\}=L^{m}$ con $m \geq 2$ ?
L'operatore di potenza permette di definire in modo espressivo il linguaggio delle stringhe di lunghezza non superiore a un intero $k$.
Esempio. Sia $\Sigma=\{a, b\}$ e $k=3$.
$$
\begin{aligned}
L &=\left\{w \in \Sigma^{*}|| w \mid \leq 3\right\} \\
&=\{\epsilon, a, b, a a, a b, b a, b b, a a a, a a b, a b a, a b b, b a a, b a b, b b a, b b b\}
\end{aligned}
$$
Altre espressioni per L:
$$
\begin{aligned}
&L=\Sigma^{0} \cup \Sigma^{1} \cup \Sigma^{2} \cup \Sigma^{3} \\
&L=\{\epsilon, a, b\}^{3} \\
&L^{\prime}=\left\{w \in \Sigma^{*}|1 \leq| w \mid \leq 3\right\}=\{a, b\}\{\epsilon, a, b\}^{2}
\end{aligned}
$$
\subsubsection{Operazioni sui linguaggi - operazioni insiemistiche}
I linguaggi sono insiemi. Quindi le operazioni insiemistiche di unione, intersezione, differenza, complemento sono definite anche per i linguaggi in quanto insiemi di stringhe;
come pure le relazioni di inclusione $(\subseteq)$, di inclusione stretta (C) e di eguaglianza (=).
\subsubsection{Operazioni sui linguaggi - differenza}
Se $L_{1}$ ed $L_{2}$ sono due linguaggi su un alfabeto $\Sigma$, la differenza $L_{1}-L_{2}$ di $L_{1}$ ed $L_{2}$ è
$$
L_{1}-L_{2}=\left\{w \in L_{1} \mid w \notin L_{2}\right\} .
$$
Notazione alternativa: $L_{1} \backslash L_{2}$.
$$
\begin{aligned}
&\Sigma=\{a, b, c\} \\
&L_{1}=\left\{\left.x|| x\right|_{a}=|x|_{b}=|x|_{c} \geq 0\right\} \\
&L_{2}=\left\{x \mid\left(|x|_{a}=|x|_{b} \geq 0\right) \wedge\left(|x|_{c}=1\right)\right\} \\
&L_{1}-L_{2}=\{\epsilon\} \cup\left\{\left.x|| x\right|_{a}=|x|_{b}=|x|_{c} \geq 2\right\} \\
&L_{2}-L_{1}=\left\{\left.x|| x\right|_{a}=|x|_{b} \neq|x|_{c}=1\right\}
\end{aligned}
$$
\subsubsection{Operazioni sui linguaggi - complemento}
Per parlare del complemento di un linguaggio (su un alfabeto $\Sigma)$ si deve introdurre un linguaggio universale, definito come l'insieme di tutte le stringhe su un alfabeto $\Sigma$.

L'insieme $\Sigma^{*}$ di tutte le stringhe su un alfabeto $\Sigma$ è l'unione di tutte le potenze dell'alfabeto.
$$
L_{\text {universale }}=\Sigma^{*}=\cup_{n \geq 0} \Sigma^{n}
$$
Se $L \subseteq \Sigma^{*}$, il complemento $\bar{L}$ di $L$ è
$$
L=\Sigma^{*}-L=\left\{w \in \Sigma^{*} \mid w \notin L\right\}
$$
Notazione alternativa: $\neg L$.
Esempio. Alfabeto: $\{a, b\}$
Linguaggio $L=\left\{w \in\{a, b\}^{*} \mid\right.$ la prima lettera di $w$ è $\left.b\right\}$ L: ?

$\bar{L}$: insieme delle stringhe su $\{a, b\}$ che non iniziano con $b$. N.B.:NON insieme stringhe che iniziano con a (es. stringa vuota $\epsilon \in \bar{L})$
II complemento di un linguaggio finito è sempre infinito.
Esempio. Alfabeto: $\{a, b\}$
Linguaggio: insieme delle stringhe di una qualsiasi lunghezza tranne che due.
$$
\bar{\{a, b\}^{2}}=\{\epsilon\} \cup\{a, b\} \cup\left(\cup_{n \geq 3}\{a, b\}^{n}\right)
$$
Non è detto però che il complemento di un linguaggio infinito sia finito.
Esempio. Alfabeto: $\{a\}$
$$
\begin{aligned}
&L=\left\{a^{2 n} \mid n \geq 0\right\} \\
&\bar{L}=\left\{a^{2 n+1} \mid n \geq 0\right\}
\end{aligned}
$$
\subsubsection{Operazioni sui linguaggi - star di Kleene}
A eccezione dell'operazione di complemento, le operazioni finora viste non permettono una descrizione finita di linguaggi infiniti. Questo è possibile mediante l'operazione star.
Chiusura di Kleene (o Kleene star o star o stella di Kleene)
Definizione
La chiusura di Kleene (o Kleene star o star) di un linguaggio L è l'unione di tutte le potenze del linguaggio:
$$
L^{*}=\bigcup_{n \in \mathbb{N}, n \geq 0} L^{n}
$$

Nota. $L^{*}$ è il linguaggio delle stringhe ottenute concatenando un numero qualsiasi di stringhe di $L$ :
$$
L^{*}=\left\{w_{1} w_{2} \ldots w_{k} \mid k \geq 0, w_{i} \in L, 1 \leq i \leq k\right\}
$$
Nota. Se $k=0, w_{1} w_{2} \ldots w_{k}=\epsilon$ è la stringa vuota.

Esempio. Dato $L=\{a b, b a\}$, ogni stringa non vuota in $L^{*}$ è la concatenazione di parole uguali ad $a b$ o a ba.
$$
L^{*}=\left\{(a b)^{n_{1}}(b a)^{m_{1}} \cdots(a b)^{n_{h}}(b a)^{m_{h}} \mid h \geq 0, n_{i}, m_{i} \geq 0, i=1, \ldots, h\right\}
$$

Nota.
$$
\emptyset^{*}=\{\epsilon\}, \quad\{\epsilon\}^{*}=\{\epsilon\}
$$
I linguaggi $\emptyset,\{\epsilon\}$ sono gli unici tali che la loro chiusura di Kleene è un linguaggio finito. Altrimenti, anche se $L$ è finito, $L^{*}$ è infinito.

- Se il linguaggio è un alfabeto $\Sigma$, la sua chiusura di Kleene $\Sigma^{*}$ è il linguaggio universale.

- Un linguaggio formale $L$ su un alfabeto $\Sigma$ è un sottoinsieme di $\Sigma^{*}: L \subseteq \sum^{*} .$

- A volte $L^{*}$ coincide con $L$.

Esempio Se $L=\left\{a^{2 n} \mid n \geq 0, n \in \mathbb{N}\right\}$, allora $L^{*}=\left\{a^{2 n} \mid n \geq 0, n \in \mathbb{N}\right\}=L .$

\subsubsection{Proprietà chiusura di Kleene}
\begin{itemize}
    \item $L \subseteq L^{*}$ (monotonicità)
    \item $\left(x \in L^{*} \wedge y \in L^{*}\right) \rightarrow x y \in L^{*}$ (chiusura rispetto alla concatenazione)
    \item $\left(L^{*}\right)^{*}=L^{*}$ (idempotenza)
    \item $\left(L^{*}\right)^{R}=\left(L^{R}\right)^{*}($ commutatività di star e riflessione $)$
\end{itemize}
Esempio Se $L=\left\{a^{2 n} \mid n \geq 0, n \in \mathbb{N}\right\}$, allora $L=\{\text { aa }\}^{*}$. Quindi
$$
\begin{aligned}
L^{*} &=\left\{a^{2 n} \mid n \geq 0, n \in \mathbb{N}\right\}^{*} \\
&=\left(\{a a\}^{*}\right)^{*} \\
&=\{a a\}^{*}=L
\end{aligned}
$$
per la proprietà di idempotenza.
\subsubsection{Identificatori}
Molti linguaggi di programmazione assegnano nomi o identificatori agli oggetti (variabili, sottoprogrammi, ecc.) utilizzati.
Una regola comune a molti linguaggi dice che un identificatore è una stringa che inizia con una lettera in $\{A, B, \ldots, Z\}$ seguita da un numero qualsiasi di lettere e cifre in $\{0,1, \ldots, 9\}$.
Esempio SOMMA32A35.

Definiti gli alfabeti
$$
\Sigma_{A \ell}=\{A, B, \ldots, Z\}, \quad \Sigma_{N}=\{0,1, \ldots, 9\}
$$
il linguaggio $I \subseteq\left(\Sigma_{A \ell} \cup \Sigma_{N}\right)^{*}$ degli identificatori risulta:
$$
I=\Sigma_{A \ell}\left(\Sigma_{A \ell} \cup \Sigma_{N}\right)^{*}
$$

Sia $I_{5}$ il linguaggio degli identificatori di lunghezza al più $5 .$ Posto $\Sigma=\Sigma_{A \ell} \cup \Sigma_{N}$, risulta
$$
\begin{aligned}
I_{5} &=\Sigma_{A \ell}\left(\Sigma^{0} \cup \Sigma^{1} \cup \Sigma^{2} \cup \Sigma^{3} \cup \Sigma^{4}\right) \\
&=\Sigma_{A \ell}\left(\{\epsilon\} \cup \Sigma \cup \Sigma^{2} \cup \Sigma^{3} \cup \Sigma^{4}\right) \\
&=\Sigma_{A \ell}(\{\epsilon\} \cup \Sigma)^{4}
\end{aligned}
$$

\subsubsection{Chiusura positiva o croce}
Una operazione utile (ma non indispensabile) è la chiusura positiva (o croce).

Per un linguaggio L sull'alfabeto $\Sigma$, definiamo
$$
L^{+}=\bigcup_{n \in \mathbb{N}, n>0} L^{n}
$$
La chiusura positiva si distingue dalla chiusura di Kleene perché nell'unione non compare la potenza di $L$ con esponente zero $L^{0}=\{\epsilon\}$.
Valgono le relazioni
$$
L^{+} \subseteq L^{*}
$$
$\epsilon \in L^{+}$se e solo se $\epsilon \in L$
$$
L^{+}=L L^{*}=L^{*} L
$$

Esempio. Dato $L=\{a b, b a\}$, ogni stringa in $L^{+}$è la concatenazione di parole uguali ad $a b$ o a ba.
$$
L^{+}=\left\{(a b)^{n_{1}}(b a)^{m_{1}} \cdots(a b)^{n_{h}}(b a)^{m_{h}} \mid h>0, n_{i}, m_{i} \geq 0, \exists j n_{j}+m_{j} \neq 0\right\}
$$

Esempio.
$$
\{\epsilon, a a\}^{+}=\left\{a^{2 n} \mid n \geq 0, n \in \mathbb{N}\right\}=\{a a\}^{*}
$$
Esempio. Le stringhe di lunghezza almeno quattro:
$$
\Sigma^{4} \Sigma^{*}=\left(\Sigma^{+}\right)^{4}
$$

\subsubsection{Quoziente}
Le operazioni sui linguaggi finora viste non possono essere utilizzate per accorciare le stringhe del linguaggio/dei linguaggi su cui operano.
L'operazione di quoziente (destro) accorcia una stringa del primo linguaggio cancellandone un suffisso appartenente al secondo.

Il quoziente (destro) di L' rispetto ad L" è definito come
$$
\begin{aligned}
L=L^{\prime}\left(L^{\prime \prime}\right)^{-1} &=\left\{y \mid\left(x=y z \in L^{\prime}\right) \wedge\left(z \in L^{\prime \prime}\right)\right\} \\
&=\left\{y \mid \text { esiste } z \in L^{\prime \prime} \text { tale che } y z \in L^{\prime}\right\}
\end{aligned}
$$
Notazione alternativa: $L^{\prime} /{ }_{D} L^{\prime \prime}$

Esempio. Siano
$$
L^{\prime}=\left\{a^{2 n} b^{2 n} \mid n \in \mathbb{N}, n>0\right\}, \quad L^{\prime \prime}=\left\{b^{2 n+1} \mid n \in \mathbb{N}, n \geq 0\right\}
$$
$$
\begin{gathered}
L^{\prime}\left(L^{\prime \prime}\right)^{-1}=\left\{a^{r} b^{s} \mid r, s \in \mathbb{N},(r \geq 2, \text { pari }) \wedge(1 \leq s<r, s \text { dispari })\right\} \\
L^{\prime \prime}\left(L^{\prime}\right)^{-1}=\emptyset
\end{gathered}
$$

Esiste un'operazione duale, il quoziente sinistro $L^{\prime} / s L^{\prime \prime}$ che accorcia una stringa del primo linguaggio cancellandone un prefisso appartenente al secondo.
$$
L=\left(L^{\prime \prime}\right)^{-1} L^{\prime}=\left\{z \mid\left(x=y z \in L^{\prime}\right) \wedge\left(y \in L^{\prime \prime}\right)\right\}
$$
Notazione alternativa: $L^{\prime} / { }_{S} L^{\prime \prime}$

\let\cleardoublepage\clearpage
